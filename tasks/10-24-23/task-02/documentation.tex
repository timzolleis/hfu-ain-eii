\documentclass[a4paper,11pt,titlepage]{article}

\usepackage{ucs}
% per input encoding kann man Umlaute direkt einsetzten, aber  dann ist man von Font des jeweiligen Rechners abh"angig. Daher mag ich es nicht!
%\usepackage[utf8x]{inputenc}
\usepackage[german,ngerman]{babel}
\usepackage{fontenc}
\usepackage[pdftex]{graphicx}
%\usepackage{latexsym}

\usepackage[pdftex]{hyperref}

\begin{document}

% hier aktuelle Uebungsnummer einfuegen
    \title{Einf\"uhrung in die Informatik\\
    Ausarbeitung \"Ubung 2}

% Namen der Bearbeiter einfuegen

    \author{hier steht dann der oder die Namen(n)}

% aktuelles Datum einfuegen

    \date{\today}

    \maketitle{\thispagestyle{plain}}


    \section{Aufgabe 1}
    Der Titel "`Aufgabe 1"' ist durch den konkreten Aufgabentitel zu ersetzen!
    Wie man mit LaTeX gut umgehen kann ist in \cite{lkurz} gut beschrieben.
    ... und auch in vielen Beispielen im Internet aufzufinden.

    \subsection{Problem}
    Das muss ich erst mal selbst formulieren, damit mir klar wurde was ich tun soll ...

    \subsection{L"osungskonzept}
    Wie gehe ich an das Problem heran ... google, freundliche Komilitonen, ...

    \subsection{konkrete L"osung}
    Wie habe ich welche Ergebnisse dann tats"achlich erzielt...

    \subsection{Tests}
    Wie kann ich sicher sein, dass meine Ergebnisse auch stimmen?


    \section{Aufgabe 2}
% hier dann die eigene Bearbeitung einfuegen
    Sollte es mal mit Mathematik im Text losgehen, dann sind besondere
    Formatierungen notwendig. So kann mit \$ als Schalter die Mathematik
    an- und ausgeschaltet werden.\\
    Z.B. $10_2$ oder $10_{16}$


    \section{Aufgabe 3}
% hier dann die eigene Bearbeitung einfuegen
    ....

% ---------


    \section{Resumee zur dieser "Ubungsaufgabe}
    Dauer f"ur
    \begin{itemize}
        \item Durchf"uhrung
        \item Dokumentation
    \end{itemize}
    Welche gro"sen Probleme waren zu l"osen?

    \begin{thebibliography}{99}
% per \bibitem werden die Eintraege hier eingefuegt und per \cite{identifier}
% im Text oben referenziert
        \bibitem{lkurz}
        Walter Schmidt, J"org Knappen, Hubert Partl, Irene Hyn:
        \LaTeXe-Kurzbeschreibung,    Version 2.3, 2003


    \end{thebibliography}
\end{document}
