\documentclass[a4paper,11pt,titlepage]{article}

\usepackage{ucs}
\usepackage[german,ngerman]{babel}
\usepackage{fontenc}
\usepackage[pdftex]{graphicx}

\usepackage[pdftex]{hyperref}

\begin{document}

    \title{Einf\"uhrung in die Informatik\\
    Ausarbeitung \"Ubung 2}

% Namen der Bearbeiter einfuegen

    \author{Tim Zolleis}

% aktuelles Datum einfuegen

    \date{\today}

    \maketitle{\thispagestyle{plain}}


    \section{Aufgabe 1 - Umrechnung zwischen Zahlensystemen}

    \subsection{Problem}
    In dieser Aufgabe sollen folgende Zahlensysteme erkannt und jeweils ineinander "ubersetzt werden.

    \begin{itemize}
        \item Dezimalsystem (10er System)
        \item Hexadezimalsystem (16er System)
        \item Bin"arsystem (2er System)
    \end{itemize}

    Konkret sind folgende Zahlen umzurechnen:

    \begin{itemize}
        \item $192_10$ (Dezimal)
        \item $0C_{16}$ (Hexadezimal)
        \item $01111110_2$ (Bin"ar)
    \end{itemize}



    Danach soll (anhand eines Beispiels aus den IPv4-Adressbereichen) die jeweils kleinste und gr"oßte darstellbare Zahl im Hexadezimalsystem und im Dezimalsystem ermittelt werden.

    \subsection{L"osungskonzept}
    Zunächst muss die Schreibweise jedes Zahlensystems verstanden werden, W"ahrend das Dezimalsystem hier keine Schwierigkeiten bereitet, da es einfach die ganzzahlige Zahl 192 darstellt, finden sich ansonsten folgende Besonderheiten: \\ \\
    \textbf{Hexadezimal}: Hier wird anstatt einer Zahl der Buchstabe C verwendet. Im Hexadezimalsystem werden nur die Zahlen von 0-9 als Ziffern dargestellt, der Rest (von einschließlich 10 bis einschließlich 15) wird mit den Buchstaben A-F dargestellt. Dies wird jeweils mal die Basis 16 hoch n gerechnet. 0C l"asst sich also als $12_{10}$ darstellen, da $12_{10} = 1*16^1 + 2*16^0 = 12_{10}$ \\ \\
    \textbf{Bin"ar}: Das Bin"arsystem (hier mit 8 Bit dargestellt) wird von rechts nach links ausgelesen, wobei sich jede Zahl "verdoppelt", da es sich hier um Zweierpotenzen handelt (2^n). Somit ist 128, $2^n$, $10000000_2$ \\ \\

    \subsection{Konkrete L"osung}

    \subsection*{Dezimal}
    Umrechnung der Dezimalzahl in andere Systeme: \\ \\
    \noindent \textbf{Hexadezimal}: 192 ist einfach in Hexadezimal umzurechnen:
    \begin{enumerate}
        \item 192 / 16 = 12 Rest 0
        \item 12 wird im Hexadezimalsystem durch den Buchstaben C repräsentiert
        \item Somit ist $192_10 = C0_{16}$
    \end{enumerate}

    \\ \\
    \noindent \textbf{Bin"ar}: Bei einer Zahl wie 192 l"asst sich dies noch relativ einfach im Kopf rechnen, da wir folgendermaßen vorgehen k"onnen:
    \begin{enumerate}
        \item $192-128$ (die gr"osste mit einer 1 darstellbare Zahl) = 64
        \item $64$ ist die zweitgr"osste mit einer 1 darstellbare Zahl, also $64-64 = 0$
        \item Somit ist $192_10$ in Bin"ar = $11000000_2$
    \end{enumerate}
    Andernfalls kann auch wie folgt vorgegangen werden:
    \begin{enumerate}
        \item 192 / 2 = 96 Rest 0
        \item 96 / 2 = 48 Rest 0
        \item 48 / 2 = 24 Rest 0
        \item 24 / 2 = 12 Rest 0
        \item 12 / 2 = 6 Rest 0
        \item 6 / 2 = 3 Rest 0
        \item 3 / 2 = 1 Rest 1
        \item 1 / 2 = 0 Rest 1
    \end{enumerate}

    \noindent Diese Zahlenfolge kann nun von rechts nach links so aufgeschrieben werden, um die entsprechende Zahl im Bin"arsystem zu erhalten.

    \subsection*{Hexadezimal}
    Umrechung der Hexadezimalzahl in andere Systeme: \\ \\
    \noindent \textbf{Dezimal}: Hier wird die Zahl einfach in das Dezimalsystem umgerechnet. Da es sich hier um die Zahl 12 handelt, lässt sich diese einfach aufschreiben:
    $0C_{16} = 12_10$ \\ \\
    \noindent \textbf{Bin"ar}:
    \begin{enumerate}
        \item 12 / 2 = 6 Rest 0
        \item 6 / 2 = 3 Rest 0
        \item 3 / 2 = 1 Rest 1
        \item 1 / 2 = 0 Rest 1
    \end{enumerate}
    Somit w"are die 4-Bit Representation von $12 1100_2$ bzw. $00001100_2$ in einem 8-Bit-System.

    \subsection{Bin"ar}
    Umrechnung der Bin"arzahl in andere Systeme: \\ \\
    \noindent \textbf{Dezimal}: Hier wird die Zahl einfach in das Dezimalsystem umgerechnet. Da es sich hier um die Zahl 126 handelt, lässt sich diese einfach aufschreiben:
    $01111110_2 = 126_10$ \\ \\


    \noindent \textbf{Hexadezimal}:
    Hier kann nach folgendem System vorgegangen werden:
    \begin{enumerate}
        \item 126 / 16 = 7 Rest 14 (14 wird als E dargestellt, also E aufschreiben)
        \item 7 / 16 = 0 Rest 7 (7 wird als 7 dargestellt, also 7 aufschreiben)
        \item Hier wird auch andersrum gelesen als geschrieben, somit $7E_{16}$
    \end{enumerate}

    \subsection{Tests}
    Da hier Umrechnungsmethoden verwendet werden, die auf der Division der jeweiligen Zahlen basieren, sind die Ergebnisse (sofern keine Rechenfehler vorliegen) immer korrekt.

    \subsection{IPv4-Adressen}
    \begin{enumerate}
        \item \textbf{H"ochstwertiges Bit muss 0 sein:} 0xxxxxxx. Somit ist die kleinste darstellbare Zahl 0 und die gr"oßte darstellbare Zahl 127. (bzw 0 und 007F im Hexadezimalsystem)
        \item  \textbf{H"ochstwertiges Bit muss 1 sein, das zweith"ochstwertige Bit muss 0 sein:} 10xxxxxx. Somit ist die kleinste darstellbare Zahl 128 und die gr"oßte darstellbare Zahl 191. (bzw 80 und BF im Hexadezimalsystem)
        \item \textbf{H"ochstwertiges Bit muss 1 sein, das zweith"ochstwertige Bit muss ebenfalls 1 sein, das dritth"ochstwertige Bit muss 0 sein}: 110xxxxx. Somit ist die kleinste darstellbare Zahl 192 und die gr"oßte darstellbare Zahl 223. (bzw C0 und DF im Hexadezimalsystem)
    \end{enumerate}


    \section{Aufgabe 2 - Gebrochenrationale Zahlen}

    \subsection{Problem}
    Zun"achst ist die gr"oßtm"ogliche Dezimalzahl (pro Zahlensystem) zu bestimmen, wenn 4 Bit f"ur Vorkomma und 4 Bit f"ur Nachkommastelle zur Verf"ugung stehen.

    Danach ist eine Konversion von Gebrochenrationalen Zahlen in die verschiedenen Zahlensysteme durchzuf"uhren.

    \subsection{L"osungskonzept}
    Hier muss verstanden werden, wie die jeweiligen Zahlensysteme die Zahlen darstellen um dann die gr"oßtm"ogliche Zahl im 4-Bit System zu ermitteln. \\ \\

    \noindent \textbf{Hexadezimal}: Da hier (wie zu Beginn beschrieben) die Zahl als Faktor $16^n$ beschrieben wird, setzt sich die gr"oßte mit 4-Bit darstellbare Zahl wie folgt zusammen: $(15 * 16^3) + (15 * 16^2) + (15 * 16^1) + (15 * 16^0)$ \\ \\
    \noindent \textbf{Bin"ar}: Da im Bin"arsystem die jeweiligen Zahlen als $2^n$ beschrieben werden, setzt sich die gr"oßte mit 4-Bit darstellbare Zahl einfach aus vier Einsen zusammen: 1111 \\ \\
    \noindent \textbf{Oktal}: Das Oktalsystem verh"alt sich gleich wie das Hexadezimalsystem, jedoch nur zur Basis 8 (hier k"onnen Zahlen von 0-7 als Faktor verwendet werden). Die gr"oßtm"ogliche 4-Bit Zahl w"are somit $(7 * 8^3) + (7 * 8^2) + (7 * 8^1) + (7 * 8^0)$ \\ \\

    \subsection{L"osung}

    \noindent \textbf{Hexadezimal}:
    Da nun bekannt ist, wie die groesste Zahl im 4-Bit System aussieht, kann dies nun vor und nach dem Komma geschrieben werden und dann in Dezimal umgerechnet werden.
    $(15 * 16^3) + (15 * 16^2) + (15 * 16^1) + (15 * 16^0) = 61440 + 3840 + 240 + 15 = 65535$ bzw. $FFFF_{16}$. Diese Zahl kann nun in Dezimal konvertiert werden:
    \begin{enumerate}
        \item 65535 kann vor dem Komma stehen gelassen werden.
        \item Die Zahlen nach dem Komma m"ussen hier dann als $15 * 16^{-n}$ geschrieben werden. Somit $(15 * 16^{-1}) + (15 * 16^{-2}) + (15 * 16^{-3}) + (15 * 16^{-3})$, was ca. 0.9999847412109375 ergibt.
        \item Somit ist die gr"oßtm"ogliche Zahl (je nach Genauigkeit der Nachkommastelle) im 8-Bit Hexadezimalsystem (mit 4 Bit f"ur die Nachkommastelle) 65535.9999847412109375
    \end{enumerate}

    \noindent \textbf{Bin"ar}:
    Da nun bekannt ist, wie die groesste Zahl im 4-Bit System aussieht, kann dies nun vor und nach dem Komma geschrieben werden und dann in Dezimal umgerechnet werden.
    $1111 = (1 * 2^3) + (1 * 2^2) + (1 * 2^1) + (1 * 2^0) = 8 + 4 + 2 + 1 = 15$ bzw. $15_{10}$. Diese Zahl kann nun in Dezimal konvertiert werden:
    \begin{enumerate}
        \item 15 kann vor dem Komma stehen gelassen werden.
        \item Die Zahlen nach dem Komma m"ussen hier dann als $1 * 2^{-n}$ geschrieben werden. Somit $(1 * 2^{-1}) + (1 * 2^{-3}) + (1 * 2^{-3}) + (1 * 2^{-4})$, was ca. 0.9375 ergibt.
        \item Somit ist die gr"oßtm"ogliche Zahl (je nach Genauigkeit der Nachkommastelle) im 8-Bit Bin"arsystem (mit 4 Bit f"ur die Nachkommastelle) 15.9375
    \end{enumerate}

    \noindent \textbf{Oktal}:
    Da nun bekannt ist, wie die groesste Zahl im 4-Bit System aussieht, kann dies nun vor und nach dem Komma geschrieben werden und dann in Dezimal umgerechnet werden.
    $7777 = (7 * 8^3) + (7 * 8^2) + (7* 8^1) * (7* 8^0) = 3584 + 448 + 56 + 7 = 4095$

    \begin{enumerate}
        \item Die 4095 kann vor dem Komma stehen gelassen werden.
        \item Die Zahlen nach dem Komma m"ussen hier dann als $7 * 8^{-n}$ geschrieben werden. Somit $(7 * 8^{-1}) + (7 * 8^{-2}) + (7 * 8^{-3}) + (7 * 8^{-4})$ was ca.999755859375 ergibt.
        \item Somit ist die gr"oßtm"ogliche Zahl (je nach Genauigkeit der Nachkommastelle) im 8-Bit Oktalsystem (mit 4 Bit f"ur die Nachkommastelle) 4095.999755859375
    \end{enumerate}

    \\ \\

    \begin{table}[h]
        \centering
        \begin{tabular}{|l|l|l|}
            \hline
            Dualsystem        & Oktalsystem & Hexadezimalsystem \\ \hline
            101101.101        & 55.5        & 2D.A              \\ \hline
            10101011.11001101 & 253.632     & AB.CD             \\ \hline
        \end{tabular}
        \label{tab:table}
    \end{table}

    \\ \\

    \noindent \textbf{Disclaimer}: Da hier nicht klar war, ob die umgerechneten Zahlen wieder zu begrenzen sind (z.B auf 8 Bit total) wurde bei der letzten Umrechnung einfach das Äquivalent der Zahl ohne Beachtung der Bitlänge angegeben.


    \section{Aufgabe 3 - Bin"are Addition / Subtraktion}

    \subsection{Problem}
    Zun"achst sollen Dezimalzahlen in das Bin"arsystem umgewandelt und dann miteinander addiert / voneinander subtrahiert werden. Hierbei sind jedoch die Besonderheiten der Verrechnung zweier Bin"arzahlen zu beachten

    \subsection{L"osungskonzept}
    Bei der Bin"arrechnung sind folgende Regeln zu beachten:
    \begin{itemize}
        \item Bei der Additionsrechnung werden die beiden Zahlen untereinander geschrieben und dann die Zahlen addiert. Im Falle 1 + 1 wird die Zahl (von rechts nach links gehend) "ubertragen.
        \item Bei der Subtraktion müssen die Zahlen zunächst auf 8-Bit erweitert werden. Die zu subtrahierden Zahl wird in ihr Zweierkomplement umgewandelt (sprich die negative Repr"asentation) und dann addiert.
    \end{itemize}
    Danach kann jede Dezimalzahl mit den in Aufgabe 1 erlernten Techniken in eine Bin"arzahl umgerechnet und entsprechend verrechnet werden

    \subsection{Konkrete L"osung}

    \noindent \textbf{Umrechnung}
    \begin{enumerate}
        \item 125 + 199 = 01111101 + 11000111
        \item 27 + 30 = 00011011 + 00011110
        \item 115 + 21 = 01110011 + 00010101
        \item 55 - 120 = 00110111 - 01111000
        \item 42 - 12 = 00101010 - 00001100
        \item 18 - 105 = 00010010 - 01101001
    \end{enumerate}
    \\ \\
    \noindent \textbf{Berechnung}
    \\ \\
    \noindent \textbf{Zweierkomplement - Beispiel}
    Um die Subtraktion zu vereinfachen wird die zu subtrahierende Zahl in ihr Zweierkomplement umgewandelt.
    \begin{enumerate}
        \item Die Bin"arrepr"asentation wird invertiert (0 wird zu 1 und 1 wird zu 0)
        \item 1 wird addiert
        \item Beispiel: 55-120
        \item 55 kann als 00110111 dargestellt werden und stehen gelassen werden
        \item 120 kann als 01111000 dargestellt werden, wird aber in das Zweierkomplement umgewandelt: 10001000
    \end{enumerate}

    \subsubsection*{Disclaimer}
    Die Nachfolgenden Subtraktionen wurden nach dem oben beschriebenen Schema durchgef"uhrt. Um das Lesen der Ergebnisse zu vereinfachen, wurde bei negativen Ergebnissen wieder das Zweierkomplement gebildet (nach gleichem Schema) um die positive Repr"asentation der Zahl zu erhalten (sprich 65 statt 191)
    Um die Negativit"at der Zahl zu kennzeichnen, wurde ein Minuszeichen vor die Zahl gesetzt. Dies ist ist bei einer bin"aren Schreibweise normalerweise \textbf{nicht} der Fall.
    \begin{enumerate}
        \item 125 + 199 = 01111101 + 11000111 = 0101000100 = 324
        \item 27 + 30 = 00011011 + 00011110 = 00111001 = 57
        \item 115 + 21 = 01110011 + 00010101 = 10001000 = 136
        \item 55 - 120 = 00110111 - 01111000 = -01000001 = -65
        \item 42 - 12 = 00101010 - 00001100 = 00011110 = 30
        \item 18 - 105 = 00010010 - 01101001 = -01001011 = -75
    \end{enumerate}
    \\


    Hier ist es n"otig, die 8-Bit Begrenzung aufgrund des "Ubertrags zu beachten, da sonst die Subtraktion unter umst"anden falsche Ergebnisse liefert.

% ---------


    \section{Resumee zur dieser "Ubungsaufgabe}
    Dauer f"ur
    \begin{itemize}
        \item Durchf"uhrung 30min
        \item Dokumentation 1h
    \end{itemize}
    Welche gro"sen Probleme waren zu l"osen?
    \begin{enumerate}
        \item Die Umrechnung von Dezimalzahlen in Bin"arzahlen + Beachtung der Nachkommastellen bei Gebrochenrationalen Zahlen
        \item Die Addition / Subtraktion von Bin"arzahlen
    \end{enumerate}

\end{document}
