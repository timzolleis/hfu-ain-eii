\documentclass[a4paper,11pt,titlepage]{article}

\usepackage{ucs}
\usepackage[german,ngerman]{babel}
\usepackage{fontenc}
\usepackage[pdftex]{graphicx}
\usepackage[pdftex]{hyperref}
\usepackage{listings}

\begin{document}

    \title{Einf\"uhrung in die Informatik\\
    Ausarbeitung \"Ubung 6}


    \author{Tim Zolleis}

    \date{\today}

    \maketitle{\thispagestyle{plain}}


    \section{Aufgabe 1 - Neuronales Netz}

    \subsection{Problem}
    Zum Trainieren des neuronalen Netzes m"ussen zun"achst die folgenden Vorraussetzungen erf"ullt sein:
    \begin{enumerate}
        \item Installation und Setup von Jupiter Notebook
        \item Installation von Pytorch
    \end{enumerate}

    \subsection{L"osungskonzept}
    Nach der Installation vom Jupyter Notebook und Pytorch m"ussen zun"achst die Trainings und Testdatens"atze heruntergeladen werden
    \begin{lstlisting}[language=Python]
        transform = transforms.Compose(
    [transforms.ToTensor(),
     transforms.Normalize((0.5,), (0.5,))])

    # Loading the training dataset
    trainset = torchvision.datasets.MNIST(root='./data', train=True, download=True, transform=transform)
    trainloader = torch.utils.data.DataLoader(trainset, batch_size=64, shuffle=True)

    # Loading the test dataset
    testset = torchvision.datasets.MNIST(root='./data', train=False, download=True, transform=transform)
    testloader = torch.utils.data.DataLoader(testset, batch_size=64, shuffle=False)
    \end{lstlisting}

    Anschließend definieren wir das neuronale Netz mit 3 Layern, wobei der erste Layer 128 Neuronen hat, der zweite 64 und der dritte 10.
    \begin{lstlisting}[language=Python]
    class MNet(nn.Module):
        def __init__(self):
            super(MNet, self).__init__()
            # First layer: 784 input features (28x28 pixels), 128 output features
            self.fc1 = nn.Linear(784, 128)

            # Second layer: 128 input features, 64 output features
            self.fc2 = nn.Linear(128, 64)

            # Output layer: 64 input features, 10 output features (for 10 classes)
            self.fc3 = nn.Linear(64, 10)

        def forward(self, x):
            # Flatten the image
            x = x.view(-1, 28*28)

            # Apply ReLU activation function after each hidden layer
            x = F.relu(self.fc1(x))
            x = F.relu(self.fc2(x))

        # No activation function needed for the output layer
        x = self.fc3(x)

        # Applying log_softmax for classification purposes
        return F.log_softmax(x, dim=1)
    \end{lstlisting}
    F"ur die beiden versteckten Layer wird die ReLU Aktivierungsfunktion verwendet, f"ur den Output Layer die log\_softmax Funktion.

    Zum Training wird noch ein Optimizer und ein Loss definiert. Hier wird der Adam Optimizer und der CrossEntropyLoss verwendet.
    \begin{lstlisting}[language=Python]
        net = MNet()
        criterion = nn.CrossEntropyLoss()
        optimizer = optim.Adam(net.parameters(), lr=0.001)
    \end{lstlisting}

    \subsection{Training}
    Nun kann das Netz trainiert werden. Hierf"ur wird zun"achst die Anzahl der Epochen festgelegt, in diesem Fall 20.
    \begin{lstlisting}[language=Python]
        num_epochs = 20

    with open("run_log.txt", "w") as log_file:
        for epoch in range(num_epochs):
            running_loss = 0.0
            for i, data in enumerate(trainloader, 0):
                inputs, labels = data

                optimizer.zero_grad()

                outputs = net(inputs)
                loss = criterion(outputs, labels)
                loss.backward()
                optimizer.step()

                running_loss += loss.item()
                if i % 100 == 99:
                    log_message = f'Epoch {epoch + 1}, Batch {i + 1}, Loss: {running_loss / 100:.3f}\n'
                    print(f'Epoch {epoch + 1}, Batch {i + 1}, Loss: {running_loss / 100:.3f}')
                    log_file.write(log_message)
                    running_loss = 0.0

    print('Finished Training')
    \end{lstlisting}
    Ich schreibe den Output hier in eine Datei, um sp"ater verschiedene Trainingsl"aufe vergleichen zu k"onnen.

    \subsection{Test}
    Nach dem Training kann das Netz getestet werden. Hierf"ur wird zun"achst die Anzahl der korrekt klassifizierten Bilder und die Anzahl der insgesamt klassifizierten Bilder gez"ahlt.
    \begin{lstlisting}[language=Python]
        correct = 0
        total = 0
        with torch.no_grad():
            for data in testloader:
                images, labels = data
                outputs = net(images)
                _, predicted = torch.max(outputs.data, 1)
                total += labels.size(0)
                correct += (predicted == labels).sum().item()

        accuracy = 100 * correct / total
        with open('training_log.txt', 'a') as f:
            accuracy_message = f'Accuracy of the network on the 10000 test images with Adam optimizer and 20 epochs: {accuracy:.2f} %\n'
            print(accuracy_message)
            f.write(accuracy_message)
    \end{lstlisting}

    Um hier wieder einen Vergleich zu haben, schreibe ich die Accuracy in die gleiche Datei wie den Trainingslog.
    In diesem Fall erhalte ich eine Accuracy von 97.71\%.

    Beobachtet man die Relation von Loss, Epochen und Genaugkeit, so l"asst sich ein absinken des Losses und der Genauigkeit ab ca. 10 Epochen beobachten.
    Mit 20 Epochen hat man jedoch noch ein sehr gutes Ver"haltnis von Dauer und Genauigkeit.
    Trainiert man weiter, l"auft man Gefahr das Modell zu "ubertrainieren, was zu einer schlechteren Genauigkeit f"uhren kann.

    \section{Resumee zur dieser "Ubungsaufgabe}
    Dauer f"ur
    \begin{itemize}
        \item Durchf"uhrung: 1h
        \item Dokumentation: 15min
    \end{itemize}

    \begin{thebibliography}{99}
    \end{thebibliography}
\end{document}
